\documentclass[../main.tex]{subfiles}
\begin{document}

\begin{english}
    \boldred{\textarabic{ضَمائِر}}\index{ضمير} are pronouns. See table \textarabic{\ref{table:dameer}} for a table of \textarabic{ضَمائِر}.
    
    \textarabic{ضَمير الشَأن}\index{ضمير!الشأن} is when \textarabic{هو} is used without referring to any noun, hidden or apparent. \textarabic{ضمير القِصَّة}\index{ضمير!القصة} is the same concept for \textarabic{هي}.
\end{english}

\begin{figure}[H]
\setlist{nosep, leftmargin=*, before= \vspace{-0.5\baselineskip}}
\centering
\begin{minipage}[t]{.5\textwidth}
    \centering
    \begin{forest}
        arabic,
        written order,
        [جملة اسمية
          [ضمير الشأن
            [هو]
          ]
          [الشأن
            [جملة اسمية
              [مبتدأ مرفوع
                [اللهُ]
              ]
              [خبر مرقوع
                [أَحَدٌ]
              ]
            ]
          ]
        ]
    \end{forest}
    \caption{هو اللهُ أَحَدٌ \\\textenglish{The fact of the matter is that Allah \jalla is one.}}
\end{minipage}%
\begin{minipage}[t]{.5\textwidth}
    \centering
    \begin{forest}
        arabic,
        written order,
        [جملة اسمية
          [ضمير الشأن
            [هو]
          ]
          [الشأن
            [جملة اسمية
              [مبتدأ مرفوع
                [زيدٌ]
              ]
              [خبر مرفوع
                [مُنطَلقٌ]
              ]
            ]
          ]
        ]
    \end{forest}
    \caption{هو زيدٌ مُنطَلِقٌ \\\textenglish{The fact is that Zaid is free.}}
\end{minipage}\footnotemark
\end{figure}
\footnotetext{\textenglish{Example from \cite[(\textarabic{سورة الإخلاص})]{kasshaf}}}

\begin{figure}[H]
\setlist{nosep, leftmargin=*, before= \vspace{-0.5\baselineskip}}
\centering
\begin{minipage}[t]{.5\textwidth}
    \centering
    \begin{forest}
        arabic,
        written order,
        [جملة اسمية
          [ضمير القصة
            [هي]
          ]
          [القصة
            [جملة اسمية
              [مبتدأ مرفوع
                [البَقَرَةُ]
              ]
              [خبر مرقوع
                [سَمينَةٌ]
              ]
            ]
          ]
        ]
    \end{forest}
    \caption{هي البَقَرَةُ سَمينَةٌ \\\textenglish{The fact is that the cow is fat.}}
\end{minipage}%
\begin{minipage}[t]{.5\textwidth}
    \centering
    \begin{forest}
        arabic,
        written order,
        [جملة اسمية
          [حرف مشبه بالفعل
            [إنَّـ]
          ]
          [اسم إنّ منصوب
            [ضمير القصة
              [ـها]
            ]
          ]
          [خبر إنّ مرفوع
            [القصة
              [جملة اسمية
                [مبتدأ مرفوع
                  [الإمرءةُ]
                ]
                [خبر مرفوع
                  [قائِمةٌ]
                ]
              ]
            ]
          ]
        ]
    \end{forest}
    \caption{إنّها الإمرءةُ قائِمةٌ \\\textenglish{The matter is that the lady is standing}}
\end{minipage}\footnotemark
\end{figure}
\footnotetext{\textenglish{Example derived from \cite[p.~38]{tasheel}}}

\end{document}
