\documentclass[../main.tex]{subfiles}
\graphicspath{{\subfix{../images/}}}
\begin{document}

\begin{english}
    \boldred{The} subject of \textarabic{النحو} is defined as the following
    \begin{itemize}
        \item To know the function, order, and ending of words in a sentence
        \item To understand the Quran, Hadith, and Islamic sciences
        \item To attain the pleasure of Allah ﷻ
    \end{itemize}
    This workbook is meant to be a cursory supplement to the topic and is not remotely comprehensive. There are many approaches to study and subject, and a rigorous academic approach would involve taking multiple approaches to fully understand the topic. In this workbook, the general philosophy is that through worked-out examples and exercises left for the reader to solve, a student might be able to gain some relative familiarity with the topic. This approach is akin to studying math, where the more problems a student solves the more mastery they gain over fundamentals. It is important to note that this is just one approach. A complete study of \textarabic{النحو} requires study of a formal text under the guidance of a scholar.

    I am also by no means a scholar or learned person. Much of this workbook are problems I am solving for my own sake. If there are any mistakes at all, please email me at \href{mailto:smanzoor@umich.edu}{smanzoor@umich.edu}. This text was also a foray into latex as a possible to tool to create documents such as this, using Arabic and English in one seamless document, replete with formatting. You can access the tex file or contribute \href{https://github.com/Yoyomanzoor/alnahw-bilmiran}{here}

    This text was made in reference to Nizam al-Nahw\supercite{nizam} and Tasheelul-Nahw\supercite{tasheel}, and as such I give examples based on that curriculum. I only include brief commentary where relevant.
\end{english}

\pagebreak

\end{document}