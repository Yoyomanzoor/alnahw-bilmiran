\documentclass[../main.tex]{subfiles}
\graphicspath{{\subfix{../images/}}}
\begin{document}

\begin{english}
    \boldred{The} subject of \textarabic{النحو} is defined as the following
    \begin{itemize}
        \item To know the function, order, and ending of words in a sentence
        \item To understand the Qur'ān, Hadith, and Islamic sciences
        \item To attain the pleasure of Allah \jalla
    \end{itemize}
    This book is meant to be a cursory supplement to the topic and is not remotely comprehensive. There are many approaches to study a subject, and a rigorous academic approach would involve taking multiple approaches to fully understand the topic. The general philosophy for this work is that through many worked-out examples, a student might be able to gain some familiarity with the topic and have a reference during their studies. The student is encouraged to try solving these sentences on their own before seeing the worked out examples. This approach is akin to studying math, where the more problems a student solves the more mastery they gain over fundamentals. It is important to note that this is just a supplement. A complete study of \textarabic{النحو} requires study of a formal text under the guidance of a scholar.

    I am by no means a scholar or learned person. These are examples I am parsing primarily for my own sake. If there are any mistakes, please email me at \href{mailto:smanzoor@umich.edu}{smanzoor@umich.edu}. This text was also a foray into latex as a possible to tool to create documents such as this, using Arabic and English in one seamless document, replete with formatting. You can access the tex file or contribute \href{https://github.com/Yoyomanzoor/alnahw-bilmiran}{here}.

    This text was made in reference to Nizām al-Nahw \autocite{nizam} and Tasheelul-Nahw \autocite{tasheel}, and as such I give examples based on that curriculum. I only include brief commentary where relevant. I pray Allah accepts these humble efforts.
    
    The manner of analyzing sentences in \textarabic{النحو} is a process called \textarabic{تركيب}, literally 'parsing.' This is the process of analyzing each word in a sentence and determining its role and state. This is not a vain exercise - this determination directly impacts the meaning and understanding the reader derives from the sentence. On a surface level, delineating the subject, verb, and object is the bare minimun to make sense of some sentences. A little more complex is to understand what a phrase in a sentence is referring to or what the hidden object of a verb might be.
    
    Sumeed Manzoor, Ann Arbor, Michigan, 2023.
\end{english}

\pagebreak

\end{document}
