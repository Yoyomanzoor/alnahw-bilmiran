\documentclass[../main.tex]{subfiles}
\begin{document}

\columnratio{0.7}
\colseprulecolor{Mahogany}
% \setlength{\columnseprule}{1pt}
\setlength{\columnsep}{2em}

\begin{paracol}{2}

\begin{english}
    There are four points.

    \textbf{The first point} is that useful utterances are termed \textbf{words} (\textarabic{كَلام}) and \textbf{sentences} (\textarabic{جُملَة}).

    The sentence is termed\footnotemark
    \begin{itemize}
        \item \textbf{nominal} (\textarabic{اِسمِيّة}) if it begins with a noun, for example ''Zaid is standing'' \textarabic{زيدٌ قائمٌ}.\footnotemark
        \item \textbf{verbal} (\textarabic{فِعلِيّة}) if it begins with a verb, for example ''Zaid is standing'' \textarabic{قام زيدٌ}.\footnotemark
        \item \textbf{minor} (\textarabic{صُغرى}) if it is 
        \item \textbf{major} (\textarabic{كُرى}) if it 
    \end{itemize}
    

\end{english}

\switchcolumn
وفيها أربع مسائل

المسألة الأولى أنّ اللفظَ المفيدَ يُسمّى كلامًا وجملةً، وأنّ الجملةَ تُسمّى اسمِيّة إنْ بُدِئت بِاسْمِ، نحو: (زيدٌ قائمٌ)، وفعلية إنْ بُدِئت بفعل، نحو: (قامَ زيدٌ)، وصُغْرى إنْ بُنيت على غيرها، كـ(قام أبوه) مِن قولك: (زيدٌ قامَ أبوه)، وكبرى إن كان في ضِمْنها جملة، كمجموعِ (زيدٌ قامَ أبوه). 

\end{paracol}

\begin{english}
\footnotetext{This discussion is based on }
\footnotetext{In Arabic, there is not a concept of the word 'is'.}
\footnotetext{While the translation to English is the same, the Arabic structure is different between these examples. In the first example, the noun came first. In this example, the verb came first.}
\end{english}

\end{document}