\documentclass[../main.tex]{subfiles}
\begin{document}

\begin{english}
    \boldred{These} are examples of \textarabic{تركيب} for āyāt from Qur'ān.\footnotemark
\end{english}
\footnotetext{\textenglish{Sources for \textarabic{تركيب}: \textarabic{إعراب القرآن الكريم \autocite{irab}، التحرير والتنوير \autocite{tahrir}}}}

\begin{figure}[H]
    \centering
    \begin{forest}
        arabic,
        written order,
        [جملة اسمية
            [حرف مشبه بالفعل
                [إن]
            ]
            [اسم إن منصوب
                [المتقين]
            ]
            [خبر إن مرفوع
                [متعلق
                    [جارة
                        [في]
                    ]
                    [مجرور
                        [معطوف عليه
                            [جنات]
                        ]
                        [حرف عطف
                            [و]
                        ]
                        [معطوف
                            [عيون]
                        ]
                    ]
                ]
            ]
        ]
    \end{forest}
    \caption[سورة الذاريات ١٥]{إِنَّ ٱلْمُتَّقِينَ فِى جَنَّـٰتٍ وَعُيُونٍ\footnotemark\\\textenglish{The \textit{muttaqeen} will be in gardens and springs}}
\end{figure}
\index{عطف}\index{جار مجرور}
\footnotetext{سورة الذاريات ١٥}

\begin{figure}[H]
    \centering
    \begin{forest}
        arabic,
        written order,
        [جملة اسمية
            [حرف مشبهة بالفعل واسم إن (نحن)
                [إنا]
            ]
            [جملة فعلية
                [فعل ماضي معروف مثبت\\جمع متكلم\\وفاعل (نحن) مستتر\\ومفعول به (ـك) منصوب
                    [أرسلناك]
                ]
                [متعلقان بـ(أرسل)
                    [جارة
                        [بـ]
                    ]
                    [مجرور
                        [الحق]
                    ]
                ]
                [حال منصوب
                    [معطوف عليه
                        [بشيرا]
                    ]
                    [حرف عطف
                        [و]
                    ]
                    [معطوف
                        [نذيرا]
                    ]
                ]
            ]
        ]
    \end{forest}
    \caption[سورة فاطر ٢٤]{إِنَّآ أَرْسَلْنَـٰكَ بِٱلْحَقِّ بَشِيرًۭا وَنَذِيرًۭا\footnotemark \\\textenglish{We have sent you with the truth as a bearer of good news and a warner.}}
\index{عطف}\index{جار مجرور}
\end{figure}
\footnotetext{سورة فاطر ٢٤}

\begin{figure}[H]
\centering
\begin{forest}
    arabic,
    written order,
    [جملة فعلية
        [فعل مضارع معروف مثبت\\واحد مذكر غائب\\وفاعل (هو) مستتر مرفوع
            [يعلم]
        ]
        [مفعول به منصوب
            [معطوف عليه
                [مضاف
                    [خائنة]
                ]
                [مضاف إليه
                    [الأعين]
                ]
            ]
            [حرف عطف
                [و]
            ]
            [معطوف
                [اسم موصول
                    [ما]
                ]
                [جملة فعلية
                    [فعل مضارع معروف مثبت\\واحد مؤنث غائب
                        [تخفي]
                    ]
                    [فاعل مرفوع
                        [الصدور]
                    ]
                ]
            ]
        ]
    ]
\end{forest}
\caption[سورة غافر ١٩]{يَعلَمُ خائِنةَ الأَعيُنِ وما تُخفي الصُدورُ\footnotemark\\\textenglish{He knows the treachary of the eyes and that which is concealed by the hearts.}}
\index{عطف}\index{موصولة}\index{جار مجرور}\index{مركب غير مفيدة!الإضافي}
\end{figure}
\footnotetext{سورة غافر ١٩}

\begin{figure}[H]
\setlist{nosep, leftmargin=*, before= \vspace{-0.5\baselineskip}}
\centering
\resizebox{\linewidth}{!}{%
\begin{forest}
    arabic,
    written order,
    [جملة فعلية
        [فعل مضارع منفي معروف منصوب\\جمع مذكر حاضر\\وفاعله الواو في محل رفع
            [لن تنالوا]
        ]
        [مفعول به منصوب
            [البرَّ]
        ]
        [متعلقان بـ(تنالوا)
            [جارة
                [حتى]
            ]
            [مجرور
                [جملة فعلية
                    [فعل مضارع معروف مثبت باب افعال\\منصوب بأن مضمرة\\جمع مذكر حاضر\\وفاعله الواو في محل رفع
                        [تنفقوا]
                    ]
                    [متعلقان بـ(تنفقوا)
                        [جارة
                            [من]
                        ]
                        [مجرور
                            [اسم موصول
                                [ما]
                            ]
                            [صلة
                                [جملة فعلية
                                    [فعل مضارع معروف مثبت\\جمع مذكر حاضر\\وفاعله الواو في محل رفع
                                        [تحبون]
                                    ]
                                ]
                            ]
                        ]
                    ]
                ]
            ]
        ]
    ]
\end{forest}
}
\caption[سورة آل عمران ٩٢]{لن تَنالوا البِرَّ حَتّى تُنفِقوا مِمّا تُحِبّونَ\footnotemark \\\textenglish{You all will never attain righteousness until you give from that which you love.}}
\end{figure}
\index{موصولة}\index{جار مجرور}
\footnotetext{سورة آل عمران ٩٢}

\begin{landscape}
\begin{figure}[H]
\centering
\resizebox{\linewidth}{!}{%
\begin{forest}
    arabic,
    written order,
    [جملة استأنفة
        [معطوف عليه
            [جملة فعلية
                [فعل ماض معروف مثبت باب تفعيل\\واحد مؤنث غائب
                    [ودَّتْ]
                ]
                [ذو الحال,name=S1
                    [فاعل مرفوع
                        [طائفة]
                    ]
                ]
                [متعلقان بـ(محذوف صفة الطائفة)
                    [حرف جر
                        [من]
                    ]
                    [مجرور
                        [مضاف
                            [أهل]
                        ]
                        [مضاف إليه
                            [الكتاب]
                        ]
                    ]
                ]
                [مفعول به منصوب
                    [جملة فعلية
                        [لو مصدرية
                            [لو]
                        ]
                        [فعل مضارع معروف مثبت باب تفعيل\\جمع مذكر غائب\\وفاعل الواو مرفوع
                            [يضلونـ]
                        ]
                        [مفعول به منصوب
                            [ـكم]
                        ]
                    ]
                    [الواو الحالية,name=S2
                        [و]
                    ]
                    [حال منصوب
                        [فعل مضارع معروف منفي باب تفعيل\\جمع مذكر غائب\\وفاعل الواو مرفوع
                            [ما يضلون]
                        ]
                        [مفعول به منصوب
                            [حرف الإستأناف
                                [إلا]
                            ]
                            [مستثنى مفرغ
                                [مضاف
                                    [أنفسـ]
                                ]
                                [مضاف إليه
                                    [ـهم]
                                ]
                            ]
                        ]
                    ]
                ]
            ]
        ]
        [حرف عطف
            [و]
        ]
        [معطوف
            [جملة فعلية
                [فعل مضارع معروف منفي باب ن\\جمع مذكر غائب\\وفاعل الواو مرفوع
                    [ما يشعرون]
                ]
            ]
        ]
    ]
    \draw[-] (S1) to[out=north west,in=north east] (S2);
\end{forest}
}
\caption[سورة آل عمران ٦٩]{وَدَّت طَّآئِفَةٌۭ مِّنْ أَهْلِ ٱلْكِتَـٰبِ لَوْ يُضِلُّونَكُمْ وَمَا يُضِلُّونَ إِلَّآ أَنفُسَهُمْ وَمَا يَشْعُرُونَ ٦٩\footnotemark\\\textenglish{A number of people of the Book hope to mislead you while they are misleading none but themselves}}
\end{figure}
\end{landscape}
\index{موصولة}\index{جار مجرور}\index{حال}\index{مركب غير مفيدة!الإضافي}\index{لو مصدرية}
\footnotetext{سورة آل عمران ٦٩}

\end{document}