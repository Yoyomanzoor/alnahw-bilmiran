\documentclass[../main.tex]{subfiles}
\begin{document}

\begin{figure}[H]
    \centering
    \begin{forest}
        arabic,
        written order,
        [جملة اسمية
            [حرف مشبه بالفعل
                [إن]
            ]
            [اسم إن منصوب
                [المتقين]
            ]
            [خبر إن مرفوع
                [متعلق
                    [جارة
                        [في]
                    ]
                    [مجرور
                        [معطوف عليه
                            [جنات]
                        ]
                        [حرف عطف
                            [و]
                        ]
                        [معطوف
                            [عيون]
                        ]
                    ]
                ]
            ]
        ]
    \end{forest}
    \caption[caption]{إِنَّ ٱلْمُتَّقِينَ فِى جَنَّـٰتٍ وَعُيُونٍ\footnotemark\\\textenglish{The \textit{muttaqeen} will be in gardens and springs}}
\end{figure}
\footnotetext{سورة الذاريات ١٥}

\begin{figure}[H]
    \centering
    \begin{forest}
        arabic,
        written order,
        [جملة اسمية
            [حرف مشبهة بالفعل واسم إن (نحن)
                [إنا]
            ]
            [جملة فعلية
                [فعل ماضي معروف مثبت\\جمع متكلم\\وفاعل (نحن) مستتر\\ومفعول به (ـك) منصوب
                    [أرسلناك]
                ]
                [متعلقان بـ(أرسل)
                    [جارة
                        [بـ]
                    ]
                    [مجرور
                        [الحق]
                    ]
                ]
                [حال منصوب
                    [معطوف عليه
                        [بشيرا]
                    ]
                    [حرف عطف
                        [و]
                    ]
                    [معطوف
                        [نذيرا]
                    ]
                ]
            ]
        ]
    \end{forest}
    \caption[caption]{إِنَّآ أَرْسَلْنَـٰكَ بِٱلْحَقِّ بَشِيرًۭا وَنَذِيرًۭا\footnotemark \\\textenglish{We have sent you with the truth as a bearer of good news and a warner.}}
\end{figure}
\footnotetext{سورة فاطر ٢٤}

\begin{figure}[H]
\centering
\begin{forest}
    arabic,
    written order,
    [جملة فعلية
        [فعل مضارع معروف مثبت\\واحد مذكر غائب\\وفاعل (هو) مستتر مرفوع
            [يعلم]
        ]
        [مفعول به منصوب
            [معطوف عليه
                [مضاف
                    [خائنة]
                ]
                [مضاف إليه
                    [الأعين]
                ]
            ]
            [حرف عطف
                [و]
            ]
            [معطوف
                [اسم موصول
                    [ما]
                ]
                [جملة فعلية
                    [فعل مضارع معروف مثبت\\واحد مؤنث غائب
                        [تخفي]
                    ]
                    [فاعل مرفوع
                        [الصدور]
                    ]
                ]
            ]
        ]
    ]
\end{forest}
\caption[caption]{يَعلَمُ خائِنةَ الأَعيُنِ وما تُخفي الصُدورُ\footnotemark\\\textenglish{He knows the treachary of the eyes and that which is concealed by the hearts.}}
\end{figure}
\footnotetext{سورة غافر ١٩}

\begin{figure}[H]
\setlist{nosep, leftmargin=*, before= \vspace{-0.5\baselineskip}}
\centering
\resizebox{\linewidth}{!}{%
\begin{forest}
    arabic,
    written order,
    [جملة فعلية
        [فعل مضارع منفي معروف منصوب\\جمع مذكر حاضر\\وفاعله الواو في محل رفع
            [لن تنالوا]
        ]
        [مفعول به منصوب
            [البرَّ]
        ]
        [متعلقان بـ(تنالوا)
            [جارة
                [حتى]
            ]
            [مجرور
                [جملة فعلية
                    [فعل مضارع معروف مثبت باب افعال\\منصوب بأن مضمرة\\جمع مذكر حاضر\\وفاعله الواو في محل رفع
                        [تنفقوا]
                    ]
                    [متعلقان بـ(تنفقوا)
                        [جارة
                            [من]
                        ]
                        [مجرور
                            [اسم موصول
                                [ما]
                            ]
                            [صلة
                                [جملة فعلية
                                    [فعل مضارع معروف مثبت\\جمع مذكر حاضر\\وفاعله الواو في محل رفع
                                        [تحبون]
                                    ]
                                ]
                            ]
                        ]
                    ]
                ]
            ]
        ]
    ]
\end{forest}
}
\caption[caption]{لن تَنالوا البِرَّ حَتّى تُنفِقوا مِمّا تُحِبّونَ\footnotemark \\\textenglish{You all will never attain righteousness until you give from that which you love.}}
\end{figure}
\footnotetext{سورة آل عمران ٩٢}

\end{document}