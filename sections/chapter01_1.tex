\documentclass[../main.tex]{subfiles}
\begin{document}

\subsection{خلاصة الدرس}

\begin{english}
    \boldred{\textarabic{الجملة الفعلية}} is the most common form of a sentence in Arabic. It consists of a \textarabic{فعل} (verb), a \textarabic{فاعل} (doer of the verb), and a \textarabic{مفعول} (object). The \textarabic{حال} of the \textarabic{فاعل} is always \textarabic{مرفوع} and for the \textarabic{مفعول} is always \textarabic{منصوب}. The sentence must begin with a \textarabic{فعل} to be considered a \textarabic{جملة فعلية}. The order of the \textarabic{فاعل} and \textarabic{مرفوع} does not matter.
\end{english}

\begin{figure}[H]
\setlist{nosep, leftmargin=*, before= \vspace{-0.5\baselineskip}}
\centering
\begin{forest}
    arabic,
    written order,
    [جملة فعلية
        [فعل ماضي معروف مثبت باب ن\\واحد مذكر غائب
            [أَخَذَ]
        ]
        [فاعل مرفوع
            [زَيدٌ]
        ]
        [مفعول به منصوب
            [المِفتاحَ]
        ]
    ]
\end{forest}
\caption{أَخَذَ زيدٌ المِفتاحَ \\\textenglish{Zaid took the key.}}
\label{fig:forest}
\end{figure}

\begin{figure}[H]
\setlist{nosep, leftmargin=*, before= \vspace{-0.5\baselineskip}}
\centering
\begin{forest}
    arabic,
    written order,
    [جملة فعلية
        [فعل ماضي معروف مثبت باب ن\\واحد مذكر غائب
            [أَخَذَ]
        ]
        [مفعول به منصوب
            [المِفتاحَ]
        ]
        [فاعل مرفوع
            [زَيدٌ]
        ]
    ]
\end{forest}
\caption{أَخَذَ المِفتاحَ زيدٌ \\\textenglish{Zaid took the key.}}
\end{figure}

\begin{english}
    The \textarabic{فعل} and \textarabic{فاعل} must match in gender.
\end{english}

\begin{figure}[H]
\setlist{nosep, leftmargin=*, before= \vspace{-0.5\baselineskip}}
\centering
\begin{forest}
    arabic,
    written order,
    [جملة فعلية
        [فعل ماضي معروف مثبت باب ف\\واحد مؤنث غائب
            [سَأَلَتْ]
        ]
        [فاعل مرفوع
            [إمرَءةٌ]
        ]
        [مفعول به منصوب
            [العالِمَ]
        ]
    ]
\end{forest}
\caption{سَأَلَتْ إمرَءةٌ العالِمَ \\\textenglish{The woman asked the scholar.}}
{\footnotesize \textenglish{The \textarabic{فعل} and \textarabic{فاعل} are \textarabic{مؤنث}}}
\end{figure}

\begin{english}
    The \textarabic{فاعل} can be hidden (\textarabic{مُستَتِر}) inside the \textarabic{فعل}.
\end{english}

\begin{figure}[H]
\setlist{nosep, leftmargin=*, before= \vspace{-0.5\baselineskip}}
\centering
\begin{forest}
    arabic,
    written order,
    [جملة فعلية
        [فعل ماضي معروف مثبت باب ن\\واحد مذكر غائب\\وفاعل (هو) مرفوع
            [أَخَذَ]
        ]
        [مفعول به منصوب
            [المِفتاحَ]
        ]
    ]
\end{forest}
\caption{أَخَذَ المِفتاحَ \\\textenglish{He took the key.}}
\end{figure}

% \newpage
\begin{english}
    A \textarabic{مفعول} is not necessary, depending on the \textarabic{فعل} and the context. Some \textarabic{افعال} can not take a \textarabic{مفعول}, these are called \textarabic{لازِم}. Others may or may not, depending on context. These are called \textarabic{مُتَعَدِّي}.
\end{english}

\begin{figure}[H]
\setlist{nosep, leftmargin=*, before= \vspace{-0.5\baselineskip}}
\centering
\begin{forest}
    arabic,
    written order,
    [جملة فعلية
        [فعل ماضي معروف مثبت باب ن\\جمع مذكر غائب\\وفاعل (و) مرفوع
            [خَرَجوا]
        ]
    ]
\end{forest}
\caption{خَرَجوا \\\textenglish{They left.}}
\end{figure}

% \pagebreak

\begin{english}
    The\textarabic{فعل} may be \textarabic{مُثبَت} (positive) or \textarabic{مَنفي} (negative). Take, for example, \textarabic{مَشى} (He walked) and \textarabic{ما مشى} (He did not walk). If the verb is preceded by a \textarabic{حرف} that negates it, such as \textarabic{لا} or \textarabic{ما}, it will be \textarabic{منفي}. If there is no such preceding \textarabic{حرف}, it will be \textarabic{مثبت}.
\end{english}

\begin{figure}[H]
\setlist{nosep, leftmargin=*, before= \vspace{-0.5\baselineskip}}
\centering
\begin{forest}
    arabic,
    written order,
    [جملة فعلية
        [فعل مضارع معروف منفي باب ض\\واحد مؤنث حاضر\\وفاعل (هي) مستتر مرفوع
            [لا تَضرِبينَ]
        ]
        [مفعول منصوب
            [الكُرَةَ]
        ]
    ]
\end{forest}
\caption{لا تَضرِبينَ الكُرَةَ \\\textenglish{You (f) are not hitting the ball.}}
\end{figure}

\begin{english}
    The\textarabic{فعل} can be \textarabic{مَعروف}, where the \textarabic{فاعل} is known (i.e. an active verb), or \textarabic{مَجهول}, where the \textarabic{فاعل} is unknown (i.e. a passive verb). In the case of \textarabic{فعل مجهول}, the \textarabic{مفعول} becomes the \textarabic{نائب فاعل} and becomes \textarabic{مرفوع}. The gender of the \textarabic{فعل} must match the \textarabic{نائب فاعل}.
\end{english}

\begin{figure}[H]
\setlist{nosep, leftmargin=*, before= \vspace{-0.5\baselineskip}}
\centering
\begin{minipage}[t]{.5\textwidth}
    \centering
    \begin{forest}
        arabic,
        written order,
        [جملة فعلية
            [فعل ماضي مجهول مثبت باب ن\\واحد مؤنث غائب
                [كُتِبَتْ]
            ]
        ]
    \end{forest}
    \caption{كُتِبَتْ \\\textenglish{It was written.}}
\end{minipage}%
\begin{minipage}[t]{.5\textwidth}
    \centering
    \begin{forest}
        arabic,
        written order,
        [جملة فعلية
            [فعل ماضي مجهول مثبت باب ن\\واحد مذكر غائب
                [أُخِذَ]
            ]
            [نائب فاعل مرفوع
                [المفتاحُ]
            ]
        ]
    \end{forest}
    \caption{أُخِذَ المفتاحُ \\\textenglish{The key was taken.}}
\end{minipage}
\end{figure}

% \pagebreak

\begin{english}
    The \textarabic{حال} of a \textarabic{فعل} can be \textarabic{نصب} if a \textarabic{حرف ناصبة} precedes it or it can be \textarabic{جزم} if a \textarabic{حرف جزم} precedes it. These modifiers may also change the meaning.
\end{english}

\begin{figure}[H]
\setlist{nosep, leftmargin=*, before= \vspace{-0.5\baselineskip}}
% \centering
\begin{minipage}[t]{.5\textwidth}
    \centering
    \begin{forest}
        arabic,
        written order,
        [جملة فعلية
            [فعل مضارع معروف منفي باب ض\\جمع مذكر حاضر\\ومفعول (ني) منصوب
                [لَن تَجِدوني]
            ]
        ]
    \end{forest}
    \caption{لَن تَجِدوني \\\textenglish{You all will never find me.}}
\end{minipage}%
\begin{minipage}[t]{.5\textwidth}
    \centering
    \begin{forest}
        arabic,
        written order,
        [جملة فعلية
            [فعل ماضي معروف منفي باب ض\\واحد متكلم\\ومفعول (ـك) منصوب
                [لَم أَجِدْكَ]
            ]
        ]
    \end{forest}
    \caption{لَم أَجِدْكَ \\\textenglish{I did not find you.}}
\end{minipage}
\end{figure}

% \pagebreak

\subsection{الامثال}

\begin{english}
    Below are worked examples encompassing these concepts.
\end{english}

\begin{figure}[H]
\setlist{nosep, leftmargin=*, before= \vspace{-0.5\baselineskip}}
\centering
\begin{minipage}[t]{.5\textwidth}
    \centering
    \begin{forest}
        arabic,
        written order,
        [جملة فعلية
            [فعل ماضي معروف مثبت باب ن\\واحد مذكر غائب
                [قالَ]
            ]
            [فاعل مرفوع
                [اللهُ]
            ]
        ]
    \end{forest}
    \caption{قالَ اللهُ \\\textenglish{Allah said.}}
\end{minipage}%
\begin{minipage}[t]{.5\textwidth}
    \centering
    \begin{forest}
        arabic,
        written order,
        [جملة فعلية
            [فعل ماضي معروف منفي باب ن\\واحد مؤنث غائب
                [قامَت]
            ]
            [فاعل مرفوع
                [فاطمةٌ]
            ]
        ]
    \end{forest}
    \caption{قامَت فاطمةٌ \\\textenglish{Fatima stood.}}
\end{minipage}
\end{figure}


\begin{figure}[H]
\setlist{nosep, leftmargin=*, before= \vspace{-0.5\baselineskip}}
\centering
\begin{minipage}[t]{.5\textwidth}
    \centering
    \begin{forest}
        arabic,
        written order,
        [جملة فعلية
            [فعل مضارع معروف منفي باب ف\\واحد مذكر حاضر\\وفاعل (انتَ) مستتر مرفوع,name=S1
                [a,phantom]
                [لا تَنساني,phantom]
            ]
            [لا تَنساني,no edge,name=shared]
            [مفعول به (ني) منصوب,name=S2
                [لا تَنساني,phantom]
                [a,phantom]
            ]
        ]
        \draw (S2) -- (shared);
        \draw (S1) -- (shared);
    \end{forest}
    \caption{لا تَنساني \\\textenglish{Don't forget me.}}
\end{minipage}%
\begin{minipage}[t]{.5\textwidth}
    \centering
    \begin{forest}
        arabic,
        written order,
        [جملة فعلية
            [فعل ماضي معروف مثبت باب ن\\جمع مذكر حاضر\\وفاعل (انتم) مستتر مرفوع
                [نَصَرتُم]
            ]
            [مفعول به منصوب
                [رَجُلاً]
            ]
        ]
    \end{forest}
    \caption{نَصَرتم رَجُلاً \\\textenglish{You all helped a man.}}
\end{minipage}
\end{figure}

\clearpage

\end{document}