% Sumeed Manzoor
% Al-Nahw wal-Miran
% The University of Michigan Medical School

% https://en.wikibooks.org/wiki/LaTeX/Linguistics
\documentclass[11pt,twoside]{book}
\usepackage[a4paper,width=150mm,top=25mm,bottom=25mm,bindingoffset=6mm]{geometry}

\usepackage{makeidx}
\makeindex

\usepackage[pdfauthor={Sumeed Manzoor},pdftitle={النحو والمران}]{hyperref}

\usepackage{graphicx}
\graphicspath{ {./images/} }
\usepackage{multirow}
\usepackage{multicol}
\usepackage[table,xcdraw,dvipsnames]{xcolor}
\usepackage{lscape}
\usepackage{rotating}
\usepackage[linguistics]{forest}
\usepackage{enumitem}
\usepackage[justification=centering]{caption}
\usepackage{subcaption}
\usepackage{float}
\usepackage{fancyhdr}
\usepackage{titlesec}
\usepackage[autostyle]{csquotes}
\usepackage{paracol}
\footnotelayout{m}

\usepackage{polyglossia}
\setmainlanguage[numerals=maghrib]{arabic}
\setotherlanguage{english}

% \setmainfont[Mapping=arabicdigits, Script=Arabic]{Scheherazade New}
\setmainfont[%
    Path=fonts/,
    Mapping=arabicdigits,
    Script=Arabic, 
    UprightFont=*-Regular,
    ItalicFont=*-Regular,
    BoldFont=*-Bold,
    Extension=.ttf
]{ScheherazadeNew} % Arabic font must have m, n, it, and b defined
\newfontfamily\englishfont[%
    Path=fonts/,
    Ligatures=TeX,
    UprightFont=*-Roman,
    ItalicFont=*-Italic,
    BoldFont=*-Bold,
    BoldItalicFont=*-BdIta,
    Extension=.ttf
]{C059}

\newfontfamily\thuluthII[
    Path=fonts/, 
    Script=Arabic, 
    Extension=.ttf
]{DecotypeThuluthII} % fancier Arabic font

\newcommand\boldred[1]{\textcolor{Mahogany}{\textbf{#1}}} % for quick bold+red
\newcommand\textitred[1]{\textcolor{Mahogany}{\textit{#1}}} % for quick italic+red
\newcommand\saw[0]{\textarabic{ﷺ}}
\newcommand\jalla[0]{\textarabic{ﷻ}}

\pagestyle{fancy}
\renewcommand{\sectionmark}[1]{\markright{#1}}

\usepackage[
backend=biber,
style=mla,
sorting=anyt,
language=auto,
autolang=other
]{biblatex}
\addbibresource{citations.bib}
\usepackage{subfiles}

\hypersetup{
    colorlinks=true,
    linkcolor=blue
}

\titleformat{\chapter}
{\color{Mahogany}\normalfont\huge\bfseries}
{\color{Mahogany}\thechapter}{1em}{}

\title{النحو والمران\\لفصل العربية في جامعة ميشيغان}
\author{\textarabic{سميد منزور}}
\date{\today} % placeholder

\forestset{
  declare boolean={arb}{1},
  arabic/.style={
      for tree={
            parent anchor=south,
            child anchor=north,
            if n children=0{
              font=\bfseries,
              tier=terminal,
              delay={
                if arb={content={##1}}{},
              }
            }{},
        }
  },
  written order/.style={for tree=reversed},
}

\begin{document}

\frontmatter

\begin{titlepage} % Suppresses headers and footers on the title page

	\centering % Centre everything on the title page
	
	\scshape % Use small caps for all text on the title page
	
	\vspace*{\baselineskip} % White space at the top of the page
	
	%------------------------------------------------
	%	Title
	%------------------------------------------------
	
	\rule{\textwidth}{1.6pt}\vspace*{-\baselineskip}\vspace*{2pt} % Thick horizontal rule
	\rule{\textwidth}{0.4pt} % Thin horizontal rule
	
	\vspace{0.75\baselineskip} % Whitespace above the title
	
	{\LARGE النحو والمران\\لفصل العربية في جامعة ميشيغان\\} % Title
	
	\vspace{0.75\baselineskip} % Whitespace below the title
	
	\rule{\textwidth}{0.4pt}\vspace*{-\baselineskip}\vspace{3.2pt} % Thin horizontal rule
	\rule{\textwidth}{1.6pt} % Thick horizontal rule
	
	\vspace{2\baselineskip} % Whitespace after the title block
	
	%------------------------------------------------
	%	Subtitle
	%------------------------------------------------
	
	\textenglish{Arabic Grammar Through Worked Examples} % Subtitle or further description
	
	\vspace*{3\baselineskip} % Whitespace under the subtitle
	
	%------------------------------------------------
	%	Editor(s)
	%------------------------------------------------
	
	% Edited By
	
	\vspace{0.5\baselineskip} % Whitespace before the editors
	
	% {\scshape\Large John Smith \\ Jane Smith \\} % Editor list
	
	\vspace{0.5\baselineskip} % Whitespace below the editor list
	
	\textit{\textenglish{Darussalam Seminary\\\vspace{3mm}The University of Michigan\\Medical School}} % Editor affiliation
	
	\vfill % Whitespace between editor names and publisher logo
	
	%------------------------------------------------
	%	Publisher
	%------------------------------------------------
	
	% \plogo % Publisher logo
    \includegraphics[scale=0.4]{MasjidLogo}\\
    \includegraphics[scale=0.4]{UMLogo}
	
	\vspace{0.3\baselineskip} % Whitespace under the publisher logo
	
	\textenglish{2023} % Publication year
	
	% {\large publisher} % Publisher

\end{titlepage}

% \maketitle
\thispagestyle{empty} % No page number on 1st page
\vspace*{\fill}
\begin{center}
\noindent{\thuluthII{بسم الله الرحمن الرحيم}}

\vspace{10mm}

\noindent{\textenglish{\textcopyright Sumeed Manzoor all rights reserved.\\No reproduction, distribution, or derivation without explicit permission.\\Inshaallah, I will release these rights once the first draft is completed.}}

\vspace{10mm}

\itshape{\noindent{\textenglish{Please keep me, my parents, family, and teachers in your duas.}}}

% \vspace{10mm}

% \textenglish{Ann Arbor,}\\
% \textenglish{Michigan}

\end{center}
\vspace*{\fill}

\clearpage

\thispagestyle{empty}
\setcounter{tocdepth}{1}
\tableofcontents
\thispagestyle{empty}
\pagebreak
% \thispagestyle{empty}
% \listoffigures
% \listoftables

\mainmatter

\columnratio{0.7}
\colseprulecolor{Mahogany}
% \setlength{\columnseprule}{1pt}
\setlength{\columnsep}{2em}

\begin{paracol}{2}
\begin{english}
  I have summarized these simplified notes from \textarabic{قواعد الإعراب} to be easy for the students, near to those of intellect, and composed of three chapters.
\end{english}
\switchcolumn
هذه نُكْتة يسيرة اختصرتها من (قواعد الإعراب)، تسهيلًا على الطُّلّاب، وتقريبًا على أُولي الألباب، وتَنْحصِر في ثلاثةِ أبوابٍ
\end{paracol}

\chapter{الباب الأول في الجملة\hfill\textenglish{The sentence}}
\subfile{sections/babaljumal.tex}


% \chapter{المثال}

% \section{المثال من القرآن}
% \subfile{sections/chapter08_1}

% \section{المثال من الأحاديث}
% \subfile{sections/chapter08_2}


% \appendix


% \chapter{جداول متفرقة}
% \subfile{sections/app-tables}


% \chapter{جدول الصرف}
% \subfile{sections/app-sarf}

% \input{sections/appendix}

\backmatter

\printindex

\printbibheading[title={\textarabic{المصادر}}]
\printbibliography[keyword={arabic},heading=subbibliography,title={قواميس عربية}]
\printbibliography[keyword={english},heading=subbibliography,title={\hfill\textenglish{English sources}}]

\end{document}