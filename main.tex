% https://en.wikibooks.org/wiki/LaTeX/Linguistics
% \documentclass[a4paper,12pt]{article}
\documentclass[12pt,twoside]{book}
\usepackage[a4paper,width=150mm,top=25mm,bottom=25mm,bindingoffset=6mm]{geometry}
\usepackage[pdfauthor={Sumeed Manzoor},pdftitle={النحو بالمران}]{hyperref}

% \usepackage{geometry}
% \newgeometry{vmargin={30mm}, hmargin={12mm,17mm}}   % set the margins

\usepackage[dvipsnames]{xcolor} % for color library
% \usepackage{tikz}
\usepackage[linguistics]{forest}
\usepackage{enumitem}
\usepackage[justification=centering]{caption}
\usepackage{subcaption}
\usepackage{float}
\usepackage{fancyhdr}
\usepackage{titlesec}
\usepackage{hyperref}

\usepackage{polyglossia}
\setmainlanguage{arabic}
\setotherlanguage{english}

\setmainfont[Script=Arabic, Mapping=arabicdigits]{Amiri}
\newfontfamily\englishfont[Ligatures=TeX]{Amiri}

\newfontfamily\thuluthII[
    Path=fonts/, 
    Script=Arabic, 
    Extension=.ttf
]{DecotypeThuluthII} % fancier Arabic font

\newfontfamily\thuluth[
    Path=fonts/, 
    Script=Arabic, 
    Extension=.ttf
]{LotusLinotypeLight} % fancier Arabic font

% \usepackage{zref-perpage}
% \zmakeperpage{footnote}
% \textwidthfootnoterule
\newcommand\boldred[1]{\textcolor{Mahogany}{\textbf{#1}}} % for quick bold+red
\newcommand\textitred[1]{\textcolor{Mahogany}{\textit{#1}}} % for quick italic+red
% \pagestyle{headings}
\pagestyle{fancy}
\renewcommand{\sectionmark}[1]{\markright{#1}}

\usepackage[
backend=biber,
style=numeric,
sorting=none,
language=english,
autolang=other
]{biblatex}
\addbibresource{citations.bib}
\usepackage{subfiles}

\hypersetup{
    colorlinks=true,
    linkcolor=blue
}

\titleformat{\chapter}
{\color{Mahogany}\normalfont\huge\bfseries}
{\color{Mahogany}\thechapter}{1em}{}

\title{النحو بالمران\\لفصل العربية في جامعة ميشيغان}
\author{\textarabic{سميد منزور}}
\date{\today} % placeholder

% \AddToHook{shipout/background}{%
%  \put(0,-\paperheight){\tikz[overlay]\draw[line width=2pt, color=Mahogany](10pt,10pt) rectangle (\paperwidth-20pt,\paperheight-20pt);}} % makes page borders

\forestset{
  declare boolean={arb}{1},
  arabic/.style={
  %   where n children={0}{
  %     tier=word,
  %     delay={
  %       if arb={content={##1}}{},
  %     },
  %   }{}
  % },
  % alignment/.style={
      for tree={
            parent anchor=south,
            child anchor=north,
            if n children=0{
              font=\bfseries,
              tier=terminal,
              delay={
                if arb={content={##1}}{},
              }
            }{},
        }
  },
  written order/.style={for tree=reversed},
}

\begin{document}

\frontmatter

\maketitle
\thispagestyle{empty} % No page number on 1st page
\tableofcontents

\pagebreak

\listoffigures
\listoftables


\mainmatter

\chapter{المقدمة}
\subfile{sections/muqaddamah}

% \chapter{مقدمة الموضوع}
\subfile{sections/intro}

\chapter{الجمل}

\subfile{sections/chapter01}

\section{الجملة الفعلية}

\subfile{sections/chapter01_1}

\section{الجملة الاسمية}

\subfile{sections/chapter01_2}

\chapter{المركب غير مفيدة}

\subfile{sections/chapter02}

\section{المركب التوصيفي}

\subfile{sections/chapter02_1}

\section{المركب الإضافي}

\subfile{sections/chapter02_2}

\chapter{الحروف}

\section{الحروف الجارة}

\subfile{sections/chapter03_1}

\chapter{الحروف المشبهة بالفعل والافعال الناقصة}

\section{الحروف المشبهة}

\subfile{sections/chapter04_1}

\chapter{المثال}

\section{المثال من القرآن}

\subfile{sections/chapter08_1}

\section{المثال من الأحاديث}

\subfile{sections/chapter08_2}
\index{hello}
\index{الأحاديث}
\index{\textarabic{الأحاديث}}

\appendix

\chapter{جدول الصرف}

\subfile{sections/app-sarf}

\input{sections/appendix}

\backmatter

\printbibliography

\end{document}