% https://en.wikibooks.org/wiki/LaTeX/Linguistics
% \documentclass[a4paper,12pt]{article}
\documentclass[11pt,twoside]{book}
\usepackage[a4paper,width=150mm,top=25mm,bottom=25mm,bindingoffset=6mm]{geometry}
\usepackage[pdfauthor={Sumeed Manzoor},pdftitle={النحو بالمران}]{hyperref}

% \usepackage{geometry}
% \newgeometry{vmargin={30mm}, hmargin={12mm,17mm}}   % set the margins

\usepackage[dvipsnames]{xcolor} % for color library
% \usepackage{tikz}
\usepackage[linguistics]{forest}
\usepackage{enumitem}
\usepackage[justification=centering]{caption}
\usepackage{subcaption}
\usepackage{float}
\usepackage{fancyhdr}
\usepackage{titlesec}
% \usepackage{hyperref}

\usepackage{polyglossia}
\setmainlanguage{arabic}
\setotherlanguage{english}

% \setmainfont[Mapping=arabicdigits, Script=Arabic]{Scheherazade New}
\setmainfont[%
    Path=fonts/,
    Mapping=arabicdigits,
    Script=Arabic, 
    UprightFont=*-Regular,
    ItalicFont=*-Regular,
    BoldFont=*-Bold,
    Extension=.ttf
]{ScheherazadeNew} % Arabic font must have m, n, it, and b defined
\newfontfamily\englishfont[%
    Path=fonts/,
    Ligatures=TeX,
    UprightFont=*-Roman,
    ItalicFont=*-Italic,
    BoldFont=*-Bold,
    BoldItalicFont=*-BdIta,
    Extension=.ttf
]{C059}

\newfontfamily\thuluthII[
    Path=fonts/, 
    Script=Arabic, 
    Extension=.ttf
]{DecotypeThuluthII} % fancier Arabic font

\newfontfamily\thuluth[
    Path=fonts/, 
    Script=Arabic, 
    Extension=.ttf
]{LotusLinotypeLight} % fancier Arabic font

% \usepackage{zref-perpage}
% \zmakeperpage{footnote}
% \textwidthfootnoterule
\newcommand\boldred[1]{\textcolor{Mahogany}{\textbf{#1}}} % for quick bold+red
\newcommand\textitred[1]{\textcolor{Mahogany}{\textit{#1}}} % for quick italic+red
% \pagestyle{headings}
\pagestyle{fancy}
\renewcommand{\sectionmark}[1]{\markright{#1}}

\usepackage[
backend=biber,
style=numeric,
sorting=none,
language=english,
autolang=other
]{biblatex}
\addbibresource{citations.bib}
\usepackage{subfiles}

\hypersetup{
    colorlinks=true,
    linkcolor=blue
}

\titleformat{\chapter}
{\color{Mahogany}\normalfont\huge\bfseries}
{\color{Mahogany}\thechapter}{1em}{}

\title{النحو بالمران\\لفصل العربية في جامعة ميشيغان}
% \author{\textarabic{سميد منزور}}
% \date{\today} % placeholder

% \AddToHook{shipout/background}{%
%  \put(0,-\paperheight){\tikz[overlay]\draw[line width=2pt, color=Mahogany](10pt,10pt) rectangle (\paperwidth-20pt,\paperheight-20pt);}} % makes page borders

\forestset{
  declare boolean={arb}{1},
  arabic/.style={
  %   where n children={0}{
  %     tier=word,
  %     delay={
  %       if arb={content={##1}}{},
  %     },
  %   }{}
  % },
  % alignment/.style={
      for tree={
            parent anchor=south,
            child anchor=north,
            if n children=0{
              font=\bfseries,
              tier=terminal,
              delay={
                if arb={content={##1}}{},
              }
            }{},
        }
  },
  written order/.style={for tree=reversed},
}

\begin{document}

\frontmatter


\begin{titlepage} % Suppresses headers and footers on the title page

	\centering % Centre everything on the title page
	
	\scshape % Use small caps for all text on the title page
	
	\vspace*{\baselineskip} % White space at the top of the page
	
	%------------------------------------------------
	%	Title
	%------------------------------------------------
	
	\rule{\textwidth}{1.6pt}\vspace*{-\baselineskip}\vspace*{2pt} % Thick horizontal rule
	\rule{\textwidth}{0.4pt} % Thin horizontal rule
	
	\vspace{0.75\baselineskip} % Whitespace above the title
	
	{\LARGE النحو بالمران\\لفصل العربية في جامعة ميشيغان\\} % Title
	
	\vspace{0.75\baselineskip} % Whitespace below the title
	
	\rule{\textwidth}{0.4pt}\vspace*{-\baselineskip}\vspace{3.2pt} % Thin horizontal rule
	\rule{\textwidth}{1.6pt} % Thick horizontal rule
	
	\vspace{2\baselineskip} % Whitespace after the title block
	
	%------------------------------------------------
	%	Subtitle
	%------------------------------------------------
	
	\textenglish{Arabic Grammar Through Worked Examples} % Subtitle or further description
	
	\vspace*{3\baselineskip} % Whitespace under the subtitle
	
	%------------------------------------------------
	%	Editor(s)
	%------------------------------------------------
	
	% Edited By
	
	\vspace{0.5\baselineskip} % Whitespace before the editors
	
	% {\scshape\Large John Smith \\ Jane Smith \\} % Editor list
	
	\vspace{0.5\baselineskip} % Whitespace below the editor list
	
	\textit{\textenglish{Darussalam Seminary \\ The University of Michigan}} % Editor affiliation
	
	\vfill % Whitespace between editor names and publisher logo
	
	%------------------------------------------------
	%	Publisher
	%------------------------------------------------
	
	% \plogo % Publisher logo
	
	\vspace{0.3\baselineskip} % Whitespace under the publisher logo
	
	\textenglish{2023} % Publication year
	
	% {\large publisher} % Publisher

\end{titlepage}

% \maketitle
\thispagestyle{empty} % No page number on 1st page
\vspace*{\fill}
\begin{center}
\noindent{\textarabic{بسم الله الرحمن الرحيم}}

\vspace{10mm}

\itshape{\noindent{\textenglish{Please keep me, my parents, family, and teachers in your duas.}}}

\vspace{10mm}

\textenglish{Ann Arbor, Michigan}\\
\textenglish{2023}

\end{center}
\vspace*{\fill}

\clearpage

\thispagestyle{empty}
\tableofcontents
\thispagestyle{empty}
\pagebreak
\thispagestyle{empty}
\listoffigures
\listoftables

\mainmatter

\chapter{المقدمة}
\subfile{sections/muqaddamah}

% \chapter{مقدمة الموضوع}
\subfile{sections/intro}

\chapter{الجمل}

\subfile{sections/chapter01}

\section{الجملة الفعلية}

\subfile{sections/chapter01_1}

\section{الجملة الاسمية}

\subfile{sections/chapter01_2}

\section{الجملة الإنشائية}

\subfile{sections/chapter01_3}

\chapter{المركب غير مفيدة}

\subfile{sections/chapter02}

\section{المركب التوصيفي}

\subfile{sections/chapter02_1}

\section{المركب الإضافي}

\subfile{sections/chapter02_2}

\chapter{الحروف}

\section{الحروف الجارة}

\subfile{sections/chapter03_1}

\chapter{الحروف المشبهة بالفعل والافعال الناقصة}

\section{الحروف المشبهة}

\subfile{sections/chapter04_1}

\chapter{المثال}

\section{المثال من القرآن}

\subfile{sections/chapter08_1}

\section{المثال من الأحاديث}

\subfile{sections/chapter08_2}
\index{hello}
\index{الأحاديث}
\index{\textarabic{الأحاديث}}

\appendix

\chapter{جدول الصرف}

\subfile{sections/app-sarf}

\input{sections/appendix}

\backmatter

\printbibliography

\end{document}